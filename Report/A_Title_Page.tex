\begin{titlepage}
    \begin{center}
        \vspace*{0.5in}
        
        \Huge
        \textbf{Modeling Chaos}
        
        Study and Simulation 
        
        of Bifurcation Systems
        
        
        \vspace*{0.7in}
        \includegraphics[width=0.6\textwidth]{Images/sulogo.png}
        
        
        \vspace*{0.5in}
        \LARGE
        \textbf{967869}
        
        \vspace*{0.2in}
        College of Science
        
        Swansea University
        
        
        
        \vfill
        \normalsize
        \begin{multicols}{2}
        Department of Physics
        
        \columnbreak
        
        $19^{th}$ March 2020
        \end{multicols}
    \end{center}
    \hrule
    
    
    \begin{abstract}
    \begin{center}
        The purpose of this work is to present the reader with an introduction to 
        linear chaotic systems. We conduct a study of deterministic chaos, proving that the one-dimensional map
        $f^{n+1}(x_n)\;=\;\mu\; x_n\; (1\;-\;x_n)$
        meets the requirements for a chaotic system defined by Devaney. Further into the work we obtain Feigenbaums constant
        using 3 different methods, with the best value obtained being $\delta = 4.6692016$. The data has been generated using
        independently written software. The software utilises techniques from multiprocessing in order to decrease data processing
        time, thus allowing for more computationally demanding tasks to be attempted.
    \end{center}
    \end{abstract}
    
    
    
\end{titlepage}